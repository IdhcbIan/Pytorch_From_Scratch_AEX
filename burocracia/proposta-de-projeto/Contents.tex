\documentclass[a4paper,12pt]{article}
\usepackage[utf8]{inputenc}
\usepackage[brazil]{babel}
\usepackage{geometry}
\geometry{margin=2.5cm}
\usepackage{hyperref}

\title{Curso PyTorch do Zero}
\author{Docente Responsável: Prof. Lucas Pascotti Valem\\
Instituto de Ciências Matemáticas e de Computação (ICMC)}
\date{}

\begin{document}

\maketitle

\section*{Aulas}
\begin{itemize}
    \item Aula 1 - Machine Learning, introdução ao problema do pipeline, complexidade, informação.

    \item Aula 2 - Revisão matemática, funções (\( \mathbb{R}^n \to \mathbb{R}^n \)), derivadas parciais, gradiente.

    \item Aula 3 - Sistema de autograd do Torch do zero Parte 1: gradientes e backprop.

    \item Aula 4 - Sistema de autograd do Torch do zero Parte 2: classe Model, primeiro MLP e levando conceitos ao Torch.

    \item Aula 5 - Torch \texttt{nn.Module} e resolvendo MNIST (CPU, GPU).

    \item Aula 6 - CNNs.

    \item Aula 7 - Problemas comuns Parte 1: overfitting, exploding gradients, vanishing gradients.

    \item Aula 8 - Problemas comuns Parte 2: layer normalization, batch normalization, shuffle correspondence.

    \item Aula 9 - RNNs.

    \item Aula 10 - Dados: extração, contaminação, criação.

    \item Aula 11 - Embeddings e transfer learning.

    \item Aula 12 - Modelos pré-treinados.

    \item Aula 13 - Transformers: Parte 1 - LLMs, tokens, previsão do próximo token.

    \item Aula 14 - Transformers: Parte 2 - Mecanismo de atenção.

    \item Aula 15 - Transformers: Parte 3 - Arquitetura Transformer original (Google, tradução).

    \item Aula 16 - Transformers: Parte 4 - Arquiteturas Encoder Only (BERT).

    \item Aula 17 - Transformers: Parte 5 - Decoder Only, GPT-1.

    \item Aula 18 - Fronteiras da Inteligência Artificial.
\end{itemize}

\end{document}

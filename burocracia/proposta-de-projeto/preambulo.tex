%              ▄
%             ▀ ▀
% █▀█ █▀█ █▀▀ ▄▀█ █▀▄▀█ █▄▄ █  █ █   █▀▀█
% █▀▀ █▀▄ ██▄ █▀█ █ ▀ █ █▄█ █▄▄█ █▄▄ █▄▄█

%::::::::::::::::::::::::::::::::::::::::::::::::::::::::::::::::::::::::::::::::
%                            CONFIGURAÇÕES ESSENCIAIS                            
%                       Afetam o documento em sua íntegra                        
%................................................................................
\usepackage[utf8]{inputenc}		% Codificação de entrada: Padrão UTF-8, pois apoia acentuação
\usepackage[T1]{fontenc}		% Codificação de saída: Padrão T1 (8-bits), para hifenização e renderização correta de fontes
\usepackage[margin=2.5cm]{geometry}	% Define as margens da página
\usepackage[brazil]{babel}		% Para configurações dependentes da lingua
\usepackage{csquotes}			% Aspas sensíveis ao contexto
                        		% 'csquotes' trabalha conjuntamente com 'babel'
                        		% 'csquotes' É UMA DEPENDÊNCIA DO 'biblatex'


%::::::::::::::::::::::::::::::::::::::::::::::::::::::::::::::::::::::::::::::::
%                                  ESTILIZAÇÃO                                   
%                  Alterações na aparência visual do documento                   
%................................................................................
\usepackage[dvipsnames]{xcolor}	% Acesso a 68 cores nomeadas
\usepackage{moresize}		% Tamanhos de fonte adicionais
\usepackage{enumerate}		% Customize itens no ambiente 'enumerate'
\usepackage{multicol}		% Ambiente com multiplas colunas de texto
\usepackage[normalem]{ulem}	% Mais estilos de sublinhado (alinha sublinhados múltiplos)
                           	% 'normalem' impede \emph seja sblinhado em vez de itálico

%::::::::::::::::::::::::::::::::::::::::::::::::::::::::::::::::::::::::::::::::
%                              AMBIENTES FLUTUANTES                              
%              Gerenciamento de conteúdo como figuras, tabelas, etc...           
%................................................................................
\usepackage{subcaption}	% Formate multiplas imagens(tabulares) em um único ambiente de figura(tabela)
\usepackage{booktabs}	% Formate tabelas conforme convenção científica
\usepackage{float}	% Especificador de posição de flutuante [H] 'Here' (use com cautela)


%:::::::::::::::::::::::::::::::::::::::::::::::::::::::::::::::::::::::::::::::::
%                               GRAPHICOS E IMAGENS                               
%                         Geração e importação de imagens                         
%.................................................................................
% IMAGENS PROGRAMATICAS GERADAS NATIVAMENTE
\usepackage{tikz}		% Crie gráficos vetorias programaticamente
\usetikzlibrary{arrows.meta}	% Estilos adicionais de setas para o TikZ
\usepackage{pgfplots}		% Ferramenta para criação de gráficos
\pgfplotsset{compat=1.18}	% Enforça a versão 1.18 para consistência

% GERENCIAMENTO DE IMAGENS EXTERNAS
\usepackage{graphicx}		% Padrão LaTeX para inclusão de arquivos de imagen



%                            ▄
%                           ▀
% █▀▄▀█ ▄▀█ ▀█▀ █▀▀ █▀▄▀█ ▄▀█ ▀█▀ █ █▀▀ ▄▀█
% █ ▀ █ █▀█  █  ██▄ █ ▀ █ █▀█  █  █ █▄▄ █▀█
%:::::::::::::::::::::::::::::::::::::::::::::::::::::::::::::::::::::::::::::::::
%                             PACOTES DE MATEMÁTICA                              
%.................................................................................
% AMS (American Mathematical Society)
\usepackage{amssymb}	% pacote da AMS's para símbolos matemáticos
\usepackage{amsmath}	% pacote da AMS's para diagramação matemática
\usepackage{mathtools}  % Extende a funcionalidade de 'amsmath' (i.e.: a superconjunto)
\usepackage{amsthm}    	% AMS's package for creating theorem-like environments

% NICHE MATH PACKAGES
\usepackage{xfrac}	% Frações diagonais (\sfrac)
\usepackage{cancel}	% Permite "cancelar" (tachar) termos (\cancel e \cancelto)
\usepackage{bbm}	% Gambiarra para o número 1 em "blackboard bold" (use \mathbbm{1})
\usepackage[		% Controle sobre as fontes usedas em \mathbb, \mathcal, \mathfrak & \mathscr
  scr=boondox,	% 'boondox' é uma fonte cursiva que apoia letras minúsculas
  cal=esstix	% 'esstix' é uma fonte caligráfica que apoia letras minúsculas a custo de letras em negrito
]{mathalpha}


%:::::::::::::::::::::::::::::::::::::::::::::::::::::::::::::::::::::::::::::::::
%                       COMANDOS MATEMÁTICOS PERSONALIZADOS                       
%.................................................................................
% CONSTANTS
\newcommand{\e}{\mathscr{e}}		% Notação especial para o algarismo neperiano

% RELATIONS
\newcommand{\longto}{\longrightarrow}	% Atalho para a seta longa usada em funções (f: A --> B)
\newcommand*{\defeq}{\mathrel{\vcenter{\baselineskip0.5ex \lineskiplimit0pt
\hbox{\scriptsize.}\hbox{\scriptsize.}}}%
=}					% Símbolo de "definido como" (:=) compacto
\newcommand{\subnormal}{\triangleleft}	% Subgrupo normal (esquerda)
\newcommand{\supnormal}{\triangleright}	% Subgrupo normal (direita)

% LOGIC
\newcommand{\limplies}{\rightarrow}	% Conectivo condicional
\newcommand{\liff}{\leftrightarrow}	% Conectivo bicondicional 
\newcommand{\nand}{\uparrow}		% Conectivo NAND
\newcommand{\nor}{\downarrow}		% Conectivo NOR
\newcommand{\xor}{\veebar}		% Conectivo XOR 


% CARREGAS POR ÚLTIMO PARA EVITAR ERROS DE DEPENDÊNCIA
%:::::::::::::::::::::::::::::::::::::::::::::::::::::::::::::::::::::::::::::::::
%                                  BIBLIOGRAFIA                                   
%                               Gerencie referências                              
%.................................................................................
\usepackage[
	backend=biber,	% Use Biber no 'backend' (mais poderoso do que BibTex)
	style=ieee,	% Define o estilo de citação
	sorting=nyt	% 'nyt' é abreviação de 'name', 'year' e 'title'
]{biblatex}


%:::::::::::::::::::::::::::::::::::::::::::::::::::::::::::::::::::::::::::::::::
%                                   FINALIZAÇÃO                                   
%           Pacotes que modificam outros. DEVEM SER CARREGAS POR ÚLTIMO           
%.................................................................................
\usepackage{hyperref}		% Hiper-links para referências, índice e 'URL's
\hypersetup{
	colorlinks=true,	% true: links coloridos; false: links encaixotados
	allcolors=NavyBlue,	% Todos tipos de cor com a mesma cor
}

\newcommand{\customref}[2]{%	% Atalho para links com texto personalizado
	\hyperref[#2]{#1}%
}

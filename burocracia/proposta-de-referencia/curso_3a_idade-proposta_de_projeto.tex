\documentclass[a4paper,12pt]{article}
\usepackage[utf8]{inputenc}
\usepackage[brazil]{babel}
\usepackage{geometry}
\geometry{margin=2.5cm}
\usepackage{hyperref}

\title{Curso Pytorch do Zero}
\author{Docente Responsável: Prof. Lucas Pascotti Valem\\
Instituto de Ciências Matemáticas e de Computação (ICMC)}
\date{}

\begin{document}

\maketitle

\section*{Descrição da Atividade}
A atividade tem como objetivo introduzir pessoas idosas ao universo da informática, proporcionando um ambiente acolhedor. A familiarização com computadores, celulares e aplicativos é uma ferramenta importante para o fortalecimento dos laços sociais. A proposta é oferecer uma abordagem prática, leve e respeitosa com o tempo de aprendizagem, visando desenvolver confiança no uso da tecnologia.

Os idosos participantes serão incentivados a:
\begin{itemize}
    \item Aprender noções básicas de uso do computador;
    \item Navegar com segurança na internet e utilizar buscadores de informação;
    \item Criar e gerenciar contas de e-mail;
    \item Utilizar aplicativos de comunicação;
    \item Explorar redes sociais e plataformas de entretenimento de forma consciente;
    \item Compreender noções de segurança digital.
\end{itemize}

Os alunos da AEX farão atividades de monitoria durante as aulas, preparação e exposição de uma das aulas previstas no cronograma e participação presencial entre 25/04/2025 e 11/07/2025, todas as sextas-feiras, das 14h às 16h.

\section*{Grupo Social Alvo}
Pessoas com 60 anos ou mais, residentes em São Carlos/SP e região, que desejam aprender a utilizar o computador e a internet em atividades do dia a dia, e que possuem pouco ou nenhum contato com a tecnologia.

\section*{Carga Horária}
\begin{itemize}
    \item Total da atividade: \textbf{50 horas}
    \item Docente responsável: \textbf{24 horas}
    \item Vagas para alunos USP: \textbf{20}
\end{itemize}

\section*{Objetivos, Metas e Resultados Esperados}
Visa promover a inclusão digital, oferecendo habilidades tecnológicas essenciais para o cotidiano, fortalecendo a autonomia e ampliando as oportunidades sociais dos participantes.

\section*{Indicadores de Avaliação da Atividade}
Os idosos responderão a um questionário impresso ao final do curso, avaliando o desempenho dos monitores e a correspondência da atividade com suas expectativas.

\section*{Indicadores de Avaliação dos Alunos USP}
Serão considerados os critérios de:
\begin{itemize}
    \item Assiduidade;
    \item Preparo do material;
    \item Auxílio aos idosos;
    \item Condução das aulas e atividades.
\end{itemize}

\section*{Pré-requisitos}
\begin{itemize}
    \item Disponibilidade presencial todas as sextas-feiras entre 25/04 e 11/07, das 14h às 16h, no laboratório G1 da EESC;
    \item Experiência prévia com o público idoso é desejável, mas não obrigatória.
\end{itemize}

\section*{ODS (Objetivos de Desenvolvimento Sustentável)}
\begin{itemize}
    \item Educação de Qualidade
    \item Redução das Desigualdades
\end{itemize}

\section*{Metodologia, Ações e Resultados Esperados}
O curso visa capacitar os idosos no uso de tecnologias como internet, redes sociais e aplicativos, promovendo independência, acesso a serviços online e combate ao isolamento social, contribuindo para uma melhor qualidade de vida.

\section*{Bibliografia}
\begin{itemize}
    \item ANDRADE, A. M. de; RABELO, L. N.; PORTO, A. P.; GOMES, E. P.; LIMA, A. L. de. Inclusão digital na terceira idade: uma revisão de escopo. \textit{Brazilian Journal of Health Review}, v. 3, n. 2, p. 3231–3243, 2020. DOI: \url{10.34119/bjhrv3n2-164}.
    \item BOKEHI, José Raphael; ROCHA, Gian Vitor Almeida; ALVARENGA, Maria Carmen Vilas-Bôas. Informática para idosos. \textit{Interagir: pensando a extensão}, Rio de Janeiro, v. 1, n. 28, p. 88–101, 2020. DOI: \url{10.12957/interag.2019.53548}.
\end{itemize}

\section*{Corresponsáveis}
\begin{itemize}
    \item Prof.ª Kamila Rios da Hora Rodrigues (USP/ICMC)
    \item Vinicius Ferreira Galvão (Aluno de Pós-Graduação – ICMC)
\end{itemize}

\end{document}
